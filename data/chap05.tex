\chapter{实验}
\label{cha:expriment}



\section{实验设计}
\subsection{实验环境}
本文的实验环境分为以下几个部分:\\
1.其中硬件环境如表~\ref{tab:computerversion}和表~\ref{tab:gpuversion}。\\
2.其中软件环境如表~\ref{tab:software}。\\
3.开发和调试主要使用$cuda$自带的$cuda-gdb$,$gcc$和$nvcc$的优化等级都为$O3$
\begin{table}[htbp]
	\noindent
	\begin{minipage}{0.5\textwidth}
		\centering
		\caption{主机参数}
		\label{tab:computerversion}
		\begin{tabular}{p{2cm}p{2cm}}
			\toprule[1.5pt]
			硬件描述 & 版本或大小 \\
			\midrule[1pt]
			CPU & i7-7700 \\
			内存 & 16G  \\
			主频 & 3.6GHz  \\
			\bottomrule[1.5pt]
		\end{tabular}
	\end{minipage}%
	\noindent
	\begin{minipage}{0.5\textwidth}
		\centering
		\caption{GPU参数}
		\label{tab:gpuversion}
		\begin{tabular}{p{4cm}p{2cm}}
			\toprule[1.5pt]
			硬件描述 & 版本或大小 \\
			\midrule[1pt]
			GPU型号 & GTX-1080 \\
			全局内存 & 8G \\
			流处理器 & 20*128 \\
			共享内存(每个Block) & 48K \\
			寄存器(每个Block) & 64K \\
			\bottomrule[1.5pt]
		\end{tabular}
	\end{minipage}
\end{table}

\begin{table}[htb]
	\centering

	\begin{minipage}{0.5\textwidth}
		\caption{软件环境}
		\label{tab:software}
		\begin{tabular}{p{2cm}p{3cm}}
			\toprule[1.5pt]
			{\heiti 软件} & {\heiti 版本} \\\midrule[1pt]
			操作系统 & ubuntu 16.04 \\
			GUN Make & 4.1 \\
			gcc/g++ & 5.4.0 \\
			Cuda & 8.0 \\
			计算能力 & 6.1 \\
			NVCC & 8.0.61 \\
			\bottomrule[1.5pt]
		\end{tabular}
	\end{minipage}
\end{table}


\subsection{实验数据}
为了方便和其他的$Shapelet$算法及加速或并行算法比较,本文使用的UCR时间序列分类\cite{UCRArchive}。在数据集中选取其中的二分类数据作为本实验的实验数据,本文只考虑二分类问题,延展到多分类问题不在本文介绍。表~\ref{tab:dataset}是关于数据的大小、时间长度等信息的介绍。

把数据叙述一下。
下面这一部分可以用于DataSet

工业数据时间序列的重要性。
时间序列分类(TSC)问题与传统的分类问题是不同的,因为属性是有序的。 事实上,顺序是否与时间无关,实际上是无关紧要的。 重要的特征是可能存在依赖于排序的歧视性特征。 UCR时间序列分类和聚类库[21]的引入使提出时间序列分类算法的出版物数量迅速增长。 在2015年夏天之前,超过1,200人下载了UCR档案,并且它已被引用数百次。 该知识库有助于提高新TSC算法评估的质量。 大多数实验涉及评估超过四十个数据集,通常具有正确的重要性测试,大多数作者发布源代码。 这种评估和再现性的程度通常比机器学习和数据挖掘研究的大多数领域更好

%\begin{table}[htb]
%	\centering
%	\begin{minipage}[t]{0.8\linewidth} % 如果想在表格中使用脚注,minipage是个不错的办法
%		\caption{数据集}
%		\label{tab:dataset}
%		\begin{tabularx}{\linewidth}{1XX}
%			\toprule[1.5pt]
%			{\heiti 文件名} & {\heiti 描述} & {\heiti cccc}\\
%			\midrule[1pt]
%			thuthesis.ins & \LaTeX{} 安装文件,\textsc{DocStrip}\footnote{表格中的脚注} & \\
%			\bottomrule[1.5pt]
%		\end{tabularx}
%	\end{minipage}
%\end{table}

\subsection{实验方案设计}

\section{实验结果分析}

需要比价

\subsection{总体时间分析}
1.首先和其他结果进行比较(如果要和其他的准确率比较,用原始CPU的)
2.比较w=0导致准确率的变化
3.使用nvprof注明时间都花在哪了?然后在这里注明5.2.2和5.2.3会对这两点进行分析
\subsection{距离计算阶段时间分析}
这一部分是5.2.2
\subsection{最优分割点计算时间分析}
这里要分析时间和N的关系
\section{本章小结}