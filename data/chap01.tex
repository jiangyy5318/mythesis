\chapter{引言}
\label{cha:intro}

\section{研究背景和意义}

在信息时代,数据已经渗透到商业、经济和其他领域,并且逐渐取代经验和直觉成为支持和影响决策的主要因素之一~\cite{anandarajan2012business}。互联网、金融、工业、医药等领域都充斥着大量数据~\cite{zhang2016unsupervised},利用这些海量数据进行分析将会给各行各业带来商业模式的转化和传统工作方法的革新。而在这些海量数据中主要是的存在形式是时间序列,比如复杂工业装备成千上万的传感器数据~\cite{sprint2017analyzing},金融领域的股票期货价格、通胀率~\cite{chakraborti2007financial},医学领域的心电图~\cite{cammarota2010time},制造行业的传感器连续测量的数据流记录都是时间序列应用的特定示例。越来越多的学者和专业人士想通过数据挖掘的方法分析时间序列数据,目的是通过发现时间序列内在知识和信息来帮助人们更好地做出决策和提前预知风险等。数据挖掘逐步应用于时间序列,并且在部分领域(例如金融、工业)取得了一定的成果~\cite{rangaswamy2013time}。但是传统的数据挖掘方法不能完全应用于时间序列,原因有两点:时间序列表示随时间的连续测量获得的值的集合~\cite{esling2012time},属性是有序的,分析和挖掘时,必须考虑值与值之间的前后相关性;时间序列数据长度不一致,不符合传统数据挖掘方法固定特征个数的要求。时间序列普遍存在于各领域,人们需要大量时间序列中蕴藏的知识和信息来给予我们意见和指导我们进行决策,因此时间序列数据挖掘(Time Series Data Mining,~TSDM)已逐渐成为数据挖掘的重要分支之一。我们将这种试图从大量的时间序列数据中提取人们事先不知道的、与时间序列属性相关的有用信息和知识的方法称为时间序列数据挖掘。目前时间序列分类面临的主要问题有两个:一是计算复杂度高,主要源于时间序列长度过大和数量过多;二是时间序列数据具有长尾效应,样本极不均衡,甚至上万时间序列数据只有其中一条负样本,例如复杂设备故障数据。

时间序列数据挖掘的主要任务包括内容查询、异常检测、预测、聚类、分类和分割~\cite{esling2012time}。其中,时间序列分类(Time  Series Classification,~TSC)是时间序列数据挖掘的重要分支之一,是将时间序列当做输入,目的在于给时间序列赋予一个预定义的类标~\cite{wei2006semi}。时间序列分类方法又可以分为三种,分别是基于相似度的、基于模型的、基于特征的分类~\cite{xing2010brief,fakhrazarisurvey}。具体主要包括包括Shapelet方法~\cite{hills2014classification,mueen2011logical}、自回归移动平均模型~\cite{saboia1977autoregressive}、深度神经网络~\cite{wang2017time}等方法,这几种方法在时间序列数据上各有优劣。自回归移动平均模型~\cite{earnest2005using}(Auto Regressive Integrated Moving Average Model,~ARIMA),是统计模型中最常见的一种用来做时间序列预测的模型,ARIMA优点在于模型十分简单,只需要内生变量而不需要借助外生变量;ARIMA模型也存在不足的的地方:要求数据是稳定或者经过差分变化之后是稳定的,并且不能捕捉非线性模型,比如ARIMA就不用于股票时间序列数据。深度神经网络~\cite{zheng2014time}(Deep Neural Networks, DNN)是近年来比较流行的机器学习算法,DNN具有模拟复杂模型、高准确率等优点,但是也存在很多不足的地方:DNN数据每种类别需都要较多的数据,与时间序列数据长尾效应相矛盾;DNN的特征不具有可解释性,不能给予过多指导意义。Shapelet是一种基于相似度计算的子序列发现算法,能够从时间序列中提取具有高分类能力的时间序列子序列~\cite{hou2016efficient},具有高准确率、易解释,对于其他模型具有指导意义等优点,而且Shapelet可以用于时间序列的压缩。

但是也存在很大的不足,Shapelet发现的过程非常耗时。因此,很多人使用多种优化方法提高Shapelet的发现效率,总的来说优化方法可以划分为四类,分别是基于剪枝的,基于学习的,基于降维的,基于并行的。首先基于剪枝的Shapelet发现算法,主要是通过减少候选集的搜索空间即候选序列集合的大小。Random-Shapelet~\cite{renard2015random}是以一定的概率从候选序列集合中获得子集,在子集上进行Shapelet发现过程,可以减少执行时间,在子集中发现Shapelet主要基于高判别能力的Shapelet在时间序列中出现的频次比较高的原因,而采样概率是准确率和时间的权衡的结果。Yang等人~\cite{yang2016shapelet}使用增强自组织增量神经网络~\cite{furao2007enhanced}(Self-Organizing Incremental Neural Network,~SOINN)将时间序列数据集中相似的子序列用一个子序列形状来代表,减少候选序列集合,在较小的候选序列集合中发现Shapelet,从而减少了执行时间。基于学习的Shapelet发现算法,主要是通过建立一个模型$f$直接学习获得高分类能力的候选序列。Hou等人~\cite{hou2016efficient}通过结合广义特征向量方法和融合lasso正则方法来产生一个稀疏块状解。GRabocka等人~\cite{grabocka2014learning}利用神经网络的方法学习获得高判别能力的候选序列。基于降维的Shapelet发现算法,主要使用低维空间来进行时间序列表示,然后使用低维空间进行相似度计算。Fast-Shapelet~\cite{rakthanmanon2013fast}就是其中的代表,Fast-Shapelet是使用SAX方法结合随机投影~\cite{lin2007experiencing}将时间序列降维到成低维空间,然后在低维空间进行相似度搜索。基于图形处理器并行的Shapelet发现算法,对于不同候选序列对应的距离计算在图形处理器上进行并行,Chang等人\cite{chang2012efficient}利用图形处理器进行距离计算,明显减少执行时间。虽然经过这么多研究,但是Shapelet发现过程的耗时问题仍然是一个难题。

\section{研究内容和主要创新点}

本文针对Shapelet发现过程非常耗时这一点,我们利用图形处理器(Graphics Processing Unit, ~GPU)对于Shapelet发现过程进行并行加速。论文的主要研究内容如下:

1.针对Shapelet发现过程耗时这一点,提出关于解决耗时问题的并行框架,能够针对不同数据集大小$N$,不同时间序列长度$L$选择不同的并行方案,另外并行方案中使用的距离度量方法为动态时间规整(Dynamic Time Warping,~DTW)。

2.分别就Shapelet发现阶段各部分(距离计算阶段、计算最佳分割点阶段、候选序列筛选阶段)提出不同的并行解决方案。

3.运用统一计算设备架构(Compute Unified Device Architecture,~CUDA)相关技术和并行算法结合,使在有限的计算资源上尽可能进行更多的计算,使用的CUDA优化技术包括矩阵转置、合并内存访问、存储器冲突、线程束分歧、归约等。

本文针对Shapelet发现过程的耗时问题提出的一个基于DTW度量的Shapelet并行算法框架中,主要创新点包括以下三个方面:

1. Shapelet并行算法框架,可以针对大数据集进行计算,针对不同得数据集大小、时间序列长度选择不同的计算方式。

2. 在距离计算阶段,采用“重用”策略降低时间复杂度;在计算最佳分割点,使用一种启发式计算信息增益的方法来降低时间复杂度。

3. 使用定制化的矩阵转置大幅度降低大量的全局内存访存时间。

\section{论文组织架构}

本文围绕基于DTW距离度量的Shapelet并行算法进行阐述:

第二部分:详细介绍了Shapelet的相关定义及计算流程进行了详细的介绍,对于Shapelet通用算法进行了分析,指明了Shapelet发现过程耗时的原因;对于多种相似度计算/距离计算方法进行比较说明并说明了部分相似度计算/距离之间的关系;对于GPU/CUDA并行原理和本文用到的CUDA优化技术进行了详细描述。

第三部分:这里首先介绍基于DTW距离度量的Shapelet并行总体设计方案,并对执行过程和数据流向进行了简单描述;对于总体方案中各模块的职能进行了简单介绍;对于设计并行算法之前遇到的并行难点进行了识别和解决。

第四部分:本章对于并行总体方案中三个主要的模块进行算法介绍,其中包括w>0距离计算阶段并行方案,w=0距离计算阶段并行方案,最佳分割点计算阶段并行方案,分别介绍三个阶段的并行策略、算法设计、实现细节和性能考虑。并行策略是如何避免重复计算和算法如何被多线程利用。实现细节和性能考虑主要是本算法如何和CUDA技术进行结合,主要包括w>0距离计算阶段和全局内存访问以及存储体冲突结合,w=0距离计算阶段和线程束分歧以及矩阵转置结合。这里$w$是一个控制相似度匹配时弯曲程度的参数,会在~\ref{cha:chap02:dtw}介绍。

第五部分:针对论文的研究目的,设计了一系列的实验,就基于DTW度量的Shapelet并行方法的准确率、执行时间在多个数据集中进行验证,并和已有优化方法进行了比较。

第六部分:总结本文的主要工作,并提出了新的问题以及可能的研究方向。
