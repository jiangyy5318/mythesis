\chapter{引言}
\label{cha:intro}

\section{研究背景和意义}

在信息时代,数据已经渗透到商业、经济和其他领域,并且逐渐取代经验和直觉成为影响决策的主要因素。互联网、金融、工业、医药等领域都充斥着大量数据,利用这些海量数据进行分析将会给各行各业带来商业模式的转化和传统工作方法的革新。而在这些海量数据中主要是以时间序列的形式存储的,复杂工业装备成千上万的传感器数据,金融领域的股票期货价格、通胀率,医学领域的心电图,其他各领域的连续测量数据流记录都是时间序列应用的特定示例。越来越多的学者和各行业专业人士通过数据挖掘的方法分析时间序列数据,以期从中发现有价值的信息来帮助更好地做出决策,提前预知风险等。我们把这种试图从大量的时间序列数据中提取人们事先不知道的、与时间属性相关的有用信息和知识的方法称为时间序列数据挖掘。

经典机器学习方法不能使用于时间序列,时间序列和传统数据挖掘存在着不同。首先,时间序列属性是有序的,分析和挖掘时,必须考虑属性之间的前后相关性;其次时间序列数据长度不一,不能直接适用于经典机器学习方法。时间序列数据挖掘对人类社会、科技和经济的发展具有重大意义,正逐渐成为数据挖掘的研究热点之一。

虽然时间序列数据挖掘虽然逐渐应用于众多领域,并且在部分领域(例如金融、工业)取得了一定的成果,但是时间序列挖掘并没有普通数据挖掘方法取得巨大成功。究其原因,时间序列的数据规模、高维度、多变模式都是影响时间序列数据挖掘广泛应用的关键因素。充分利用时间序列数据,进一步发展时间序列挖掘方法,必然从数据中能够发现很多有益的信息,从而促进生产率的提高和收益增长。


\section{研究内容}

时间序列数据挖掘(time series data mining, TSDM)的主要任务包括内容查询、异常检、预测、聚类、分类和分割~\cite{esling2012time}。其中,时间序列分类(time series classification,TSC)是时间序列分析的重要任务之一,是将时间序列当做输入,目的在于给时间序列赋予一个预定义的类中。时间序列分类的方法主要包括决策树、近邻法、神经网络等。

目前,时间序列分类TSC的方法主要包括Shapelet方法、自回归移动平均模型、DNN等方法。这几种方法在时间序列数据上各有优劣:

自回归移动平均模型(Autoregressive Integrated Moving Average Model,ARIMA),是统计模型中最常见的一种用来做时间序列预测的模型。ARIMA优点在于模型十分简单,只需要内生变量而不需要借助外生变量。模型也存在不足的的地方:要求数据是稳定或者经过差分变化之后是稳定的,并且不能捕捉非线性模型,例如ARIMA就不用于股票时间序列数据。

深度神经网络(Deep neural networks,DNN)是近年来比较流行的机器学习算法。DNN具有高适用性、高准确率等优点。但是也存在很多不足的地方:DNN能够提取的特征有限,不能囊括足够的时间序列特性;DNN的特征不具有可解释性。

Shapelet虽然具有高准确率、易解释,对于其他模型具有指导意义等优点,而且Shapelet可以用于时间序列的压缩,但是也存在很大的不足:Shapelet训练非常耗时,这于Shapelet的高时间复杂度有关。
使用基于shapelet的方法的优点是其快速的分类时间,因为大部分计算复杂度是从训练数据中提取shapelets;这个训练数据组在分类过程中不需要





\subsection{论文的研究内容}

本文选择Shapelet作为分类方法,针对Shapelet训练非常耗时这一点进行了GPU加速。

论文的主要研究内容如下:

1. 针对Shapelet训练耗时这一点,提出关于解决耗时问题的并行解决框架。

2. 分别就Shapelet训练阶段各部分(距离计算阶段、计算信息增益阶段、计算最大信息增益阶段)提出不同的解决方案,并且根据数据集的不同,适用不同的算法。

3. 运用CUDA相关技术对于Shapelet并行算法进行优化。


\subsection{主要创新点}

本文针Shapelet并行问题,提出了一个基于DTW测度得Shapelet并行算法框架,论文得主要创新点包括以下三个方面:

1. Shapelet并行算法框架,可以针对大数据集进行计算,针对不同得数据集大小、时间序列长度选择不同的计算方式。

2. 在距离计算阶段,采用“重用”策略降低时间复杂度;在计算最佳分割点,使用一种启发式计算信息增益的方法来降低时间复杂度。

2. 使用并行的方法对于Shapelet算法进行并行,并使用Coalesced、Bank-Conflict、Reduced等技术应用到Shapelet并行计算中。


\section{论文组织架构}
本文围绕基于DTW测度的Shapelet并行算法进行阐述:

第二部分:详细阐述了Shapelet的背景知识,明确了Shapelet现在存在的问题;对于距离计算方法进行说明比较;详细介绍了Shapelet的算法流程进行了详细的介绍,并分析了通用算法。

第三部分:对于GPU并行原理、后文使用到的CUDA并行方法进行了描述,并介绍了Shapelet并行有可能遇到的难点。

第四部分:首先介绍了基于DTW测度的Shapelet并行方案的整体框架,分别就距离计算并行算法设计、$w=0$距离计算并行算法设计、启发式方法计算最佳分割点部分进行了详述,并对三部分的实现细节和性能考虑进行了分析。

第五部分:针对论文的研究目的,设计了一系列的实验,就基于DTW测度的Shapelet并行方法的准确率、执行事件在多个数据集中进行验证,并和现有方法进行了比较。

第六部分:总结本文的主要工作,并提出了新的问题以及可能的研究方向。
